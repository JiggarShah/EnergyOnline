\documentclass{scrartcl}
\usepackage[utf8]{inputenc}
\usepackage{amsmath,amssymb,amstext}
\usepackage{mathtools}

\usepackage[onehalfspacing]{setspace}

\begin{document}
	
	\title{Energy Online}
	\subtitle{}
	\author{J. Shah, M. Jordi, M. Stemmle, S. Oeschger, S. Zemljic}
	
	\maketitle
	
	\section{The Problem}
	
	40\% of the electricity in Switzerland is provided by nuclear power plants, witch have to stop producing energy in 2034. To cover demands, a lot more renewable energy must be produced. 
	Currently, energy surplus from private power plants (mainly PV) can only be sold to local energy provider for a fixed price. Combined with high entry barriers PV's are very unprofitable.
	Our solution is to incentive more private power plants by making them more profitable and flexible.
	
	\section{The Vision}
	
	Our goal is to create a Decentralised Autonomous Organisation (DAO). The DAO issues energy tokens for energy produced, witch can be traded freely through the blockchain network. The tokens can then be used to receive electricity again. The DAO uses the surplus energy in the network to trade on the energy stock market. With the profit from trading, new renewable power plants are subsidized. Members of the DAO have a right to vote on changes in the DAO's smart contracts. The DAO's behaviour is completely controlled by those smart contracts.  
	
	\section{The way there}
	
	\subsection{Initialization through ICO}
	At the beginning, there will be an "Initial Coin Offering" (ICO), to get the initial funds for the DAO and to distribute the tokens. A major part of the tokens will be sold to investors. A smaller part could be distributed for example to core developers or held back as reserves. 
	
	\subsection{Decentralised Autonomous Organisation DAO}
	A DAO's behaviour is defined by smart contracts. This includes how to invest in further power plants  and trading behaviour. Changes to these contracts can be decided by a majority of stakeholders. 
	
	
\end{document}